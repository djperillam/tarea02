\documentclass{article}
\usepackage{graphicx}
\usepackage{amsthm}
\begin{document}
\textbf{Termistor NTC 10$k\Omega$}

Se toma un termistor NTC de 10$k\omega$ aislado y conectado debidamente a un pa\
r de cables y se amarró de manera que estuviera siempre en contacto con el bulb\
o de un termómetro; esta unión se ubicó en el recipiente metálico con agua, la \
cual se calentó hasta llegar a su punto de ebullición. Los cables mencionados s\
e conectan a un multimetro con el que se midió la variación de la resistencia d\
el termistor en $k\Omega$ a medida que la temperatura aumentaba.

El comportamiento de la resistencia en función de la temperatura en escala abso\
luta esta dado por la siguiente ecuación

\[R(T)=R_0 \exp{\left[\beta\left(\frac{1}{T}-\frac{1}{T_0}\right)\right]}\]
\begin{figure}
\begin{center}
\includegraphics[scale=0.7]{thermistor.png}
\end{center}
\end{figure}
\end{document}
